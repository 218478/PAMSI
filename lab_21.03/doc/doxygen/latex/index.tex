\begin{DoxyAuthor}{Author}
Kamil Kuczaj \href{mailto:218478@student.pwr.edu.pl}{\tt 218478@student.\-pwr.\-edu.\-pl}
\end{DoxyAuthor}
\hypertarget{index_intro_sec}{}\section{Wstep}\label{index_intro_sec}
Program zostal zbudowany modulowo. W folderze inc/ znajduja sie pliki naglowkowe. Folder src/ zawiera pliki zrodlowe. W glownym folderze zbudowany zostal Makefile. Pliki obiektowe sa budowane w folderze obj/ a nastepnie linkowane do glownego folderu (prj/). Testowano przy wykorzystaniu kompilatora g++ w wersji 4.\-8.\-4 na systemie Linux Ubuntu 14.\-04.\-04 opartego o jądro 4.\-2.\-0-\/30-\/generic.\hypertarget{index_Licencja}{}\section{Licencja}\label{index_Licencja}
Program udostepniam na licencji G\-P\-Lv3.\hypertarget{index_install_sec}{}\section{Instalacja}\label{index_install_sec}
Aby zbudowac i jednoczesnie odpalic program\-: \$ make

Aby pozbyc sie plikow z koncowka $\ast$$\sim$ lub zaczynajacych sie na \#$\ast$\-: \$ make order

Aby pozbyc sie programu wykonywalnego oraz plikow obiektowych\-: \$ make clean

Aby wyswietlic pomoc do pliku Makefile\-: \$ make help 