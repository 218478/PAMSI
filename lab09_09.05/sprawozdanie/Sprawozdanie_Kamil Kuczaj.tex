\documentclass[11pt,a4paper]{article}
\usepackage[T1]{fontenc}
\usepackage[utf8]{inputenc}
\usepackage[polish]{babel}
\usepackage{amsmath}
\usepackage{amsfonts}
\usepackage{graphicx}
\usepackage[margin=0.7in]{geometry}
\author{Kamil Kuczaj}
\title{Sprawozdanie z Laboratorium 9 - Pomiar czasu wykonywania algorytmu Branch \& Bound w grafie nieskierowanym.}
\date{\today}
\begin{document}

\maketitle

\section{Wstęp}
\hspace{4ex}Postawione zadanie polegało na zmierzeniu czasu wyszukiwania najkrótszej drogi pomiędze losowo wybranymi wierczołkami w grafie nieskierowanym. Graf miał składać się z $10^1$, $10^3$, $10^5$, $10^6$, $10^9$ krawędzi. Tym razem objęto inny sposób generacji grafu niż w poprzednim ćwiczeniu - zdecydowano, że liczba wierzchołków będzie ustalona i będzie wynosić 1000. Następnie generowane jest n krawędzi, gdzie n odpowiada zadanej ilości elementów, odpowiednio $10^1$, $10^3$, $10^5$, $10^6$, $10^9$.\\\\Pomiary należało wykonać na dwóch algorytmach:
\begin{enumerate}
\item Branch \& Bound
\item Branch \& Bound + extended list
\end{enumerate}

\section{Specyfikacja komputera}

\begin{center}
	\begin{tabular}{| r | c |}
	\hline
	Wersja kompilatora \textit{g++} & 4.8.4 \\ \hline
	System & Ubuntu 14.04.4 \\ \hline
	Procesor	 & Intel Core i5 2510M 2.3 GHz \\ \hline
	Pamięć RAM & 8 GB DDR3 1600 MHz \\ \hline
	Dysk twardy & HDD (5400 obr./min) \\ \hline
	Rozmiar zmiennej \textit{int} & 4 bajty \\ \hline
	\end{tabular}
\end{center}

\section{Pomiary oraz ich interpretacja}

Wskutek złej organizacji pracy, zabrakło czasu na zaimplementowanie tego algorytmu.

\section{Wnioski}
\hspace{4ex}Branch \& Bound staje się dużo bardziej efektywny gdy dołączymy do niego listę wierzchołków odwiedzonych, tzw. \textit{extended list}.


Wnioski na podstawie literatury znalezionej w internecie oraz wykładu MIT Patricka Winstona.
\end{document}